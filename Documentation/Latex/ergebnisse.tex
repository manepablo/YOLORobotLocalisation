\section{Ergebnisse}
xy anpassen!!!
Die in Abschnitt 3.2 beschriebene Netzstruktur wird mit dem in Abschnitt 3.3 vorgestelltem Datensatz und Parametern trainiert und validiert. Die optimalen Trainingsparameter wurden dabei aus mehren Experimenten empirisch ermittelt. Dabei dient die in Abschnitt 3.1 erarbeitete \textit{Loss}-Metrik zur jeweiligen Quantisierung des Trainingserfolges. \\Abbildung \ref{lossbild} zeigt den Verlauf des \textit{Loss} über die Trainingsiterationen im 3D-Fall. Die Variable \textit{loss} steht für den loss der Trainingsdaten und \textit{val_loss} für den loss der Testdaten. Dabei fällt dieser zunächst relativ schnell und nähert sich dann XY an. Dies deutet auf ein erfolgreiches Training ohne \textit{Overfitting} hin.\\
\begin{figure}[!htb] %loss und val_loss graph
  \centering
  \includegraphics[width=6.8cm]{Abb/XY}
  \caption{Loss und Val_loss über Trainingsepochen}
  \label{lossbild}
\end{figure} 
In Abbildung werden beispielhaft einige prädizierte Bounding-Boxen der Testdaten dargestellt (blau). Zum Vergleich sind zusätzlich die manuell vorgelabelten Bounding-Boxen eingezeichnet (rot).  \label{pred_boxes}
\begin{figure}[!htb]
  \centering
  \includegraphics[width=6.8cm]{Abb/ULM_class_loadData.PNG}
  \caption{Trainierte und prädizierte Bounding-Boxen}
  \label{pred_boxes}
\end{figure} 