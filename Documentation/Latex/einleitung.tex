\section{Einleitung}

Objekterkennung ist einer der vielversprechendsten und am stärksten im Forschungsfokus stehenden Bereiche von \textit{Computer Vision} und \textit{Machine Learning} \cite{Ouyang2014}. Besonders im Bereich der medizinischen und industriellen Innovationen konnten dadurch bereits jetzt schon große Fortschritte erzielt werden  \cite{Tremblay1809, Abstract2016}. So können beispielhaft Operationsroboter stark von einer auf \textit{Deep-Learning}-Algorithmen basierenden Erkennung und Lokalisation der Operationsinstrumente profitieren \cite{Surgery1803}. Auch in der Tumordiagnostik kann so die Erkennungsgenauigkeit gegenüber einer menschlichen Klassifikation bei gleichzeitig signifikant niedrigeren Kosten verbessert werden \cite{Cruz-Roa2017}.\newline
Die Anforderungen an die Bildverarbeitung sind dabei in diesen Bereichen besonders groß. Zum einen sorgen immer hochauflösendere Bilder für enorme Datenmengen und damit Hardwareanforderungen. Zum anderen ist die Anzahl an möglichen Klassen und damit Featurekombinationen bei der Objekterkennung enorm. Aufgrund ihrer hohen Trainingsperformanz und der Fähigkeit herausragend effizient Features in Bildern zu erkennen, eignen sich \textit{Convolutional-Neural-Networks (CNN)} besonders gut als Lösungsansatz der beschriebenen Probleme \cite{Ouyang2014}.\newline
In der vorliegenden Arbeit wird ein \textit{Deep-Learning}-Modell zur Bestimmung der \textit{3D-Boundingbox} eines Roboters implementiert und evaluiert.  Dafür erfolgt anfangs die Abgrenzung und grobe Erläuterung des \textit{Deep-Learnings} (DL) im Kontext des \textit{Machine-Learnings} (ML). Der Fokus liegt dabei auf dem verwendeten \textit{DL}-Konzept der CNN's und damit verbundener Probleme und Herausforderungen. Anschließend erfolgt eine Abhandlung der für die Implementierung verwendeten Netzstruktur und Metriken. Die Implementierung wird zunächst zur Lösung eines zweidimensionalen Problems erstellt und dann anschließend auf drei Dimensionen erweitert. Die Entwicklung erfolgt in Python 3.7 unter Verwendung des Tensorflow-Frameworks in der Spider und Eclipse IDE. Die Evaluation der Implementierung geschieht anhand eines Vergleiches der geschätzten \textit{Boundingbox} mit der gelabelten \textit{Boundingbox} der Testdaten. Als Datengrundlage für Training und Evaluation dient ein synthetisch generierter und vorgelabelter Datensatz von RGB-Bildern.   