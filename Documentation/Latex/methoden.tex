\section{Methoden}

Zu Lösung der Aufgabenstellung wird die sogenannte YOLO (\textit{You Only Look Once}) Netzstruktur in Python unter Nutzung der Keras API implementiert. YOLO Netze zeichnen sich durch gute Detektionsergebnisse bei einer sehr hohen Trainings- und Klassifikationsperfomanz aus. So ist mithilfe des YOLO-Modells sogar auf vergleichsweise günstiger Hardware eine Echtzeitschätzung der Bounding-Boxen von Objekten möglich \cite{Redmon2016}. 

\subsection{Loss mittels Interception over Union}


\subsection{Netzstruktur}

@Paul: hab das Paper gefunden, von dem unsere Netzstruktur inspiriert ist, da kannst du mal einen Blick rein werfen wenn du weitere infos brauchst: \cite{Redmon2016}

\subsection{Klassendiagramm}
Falls du bock drauf hast wäre es vlt noch sinnvoll grob das Klassendiagramm mit den Funktionen zu zeigen (aus python). Aber auf keinen Fall details. aber wenn du so schon genug hast reicht das auch ohne